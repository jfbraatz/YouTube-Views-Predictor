%%%%%%%%%%%%%%%%%%%%%%%%%%%%%%%%%%%%%%%%%%%%%%%%%%%%%%%%%%%%%%%%%%%%%%%%%%%
%
% Template for a LaTex article in English.
%
%%%%%%%%%%%%%%%%%%%%%%%%%%%%%%%%%%%%%%%%%%%%%%%%%%%%%%%%%%%%%%%%%%%%%%%%%%%

\documentclass[12pt]{article}
\usepackage{fullpage,enumitem,amsmath,amsthm,amsfonts,amssymb,graphicx,float,listings}

% AMS packages:
% \usepackage{amsmath, amsthm, amsfonts}

% Theorems
%-----------------------------------------------------------------
\newtheorem{thm}{Theorem}[section]
\newtheorem{cor}[thm]{Corollary}
\newtheorem{lem}[thm]{Lemma}
\newtheorem{prop}[thm]{Proposition}
\theoremstyle{definition}
\newtheorem{defn}[thm]{Definition}
\theoremstyle{remark}
\newtheorem{rem}[thm]{Remark}

% Shortcuts.
% One can define new commands to shorten frequently used
% constructions. As an example, this defines the R and Z used
% for the real and integer numbers.
%-----------------------------------------------------------------
\def\RR{\mathbb{R}}
\def\ZZ{\mathbb{Z}}

% Similarly, one can define commands that take arguments. In this
% example we define a command for the absolute value.
% -----------------------------------------------------------------
\newcommand{\abs}[1]{\left\vert#1\right\vert}

% Operators
% New operators must defined as such to have them typeset
% correctly. As an example we define the Jacobian:
% -----------------------------------------------------------------
\DeclareMathOperator{\Jac}{Jac}

%-----------------------------------------------------------------
\title{CS 221 Project Proposal}
\author{Jon Braatz \& Lance Lamore}

\begin{document}
\maketitle

\abstract{We propose to create a model for predicting YouTube views for a video
  posted to an uploader's channel based on a canditate video title, thumbnail image, and statistics from
  previously uploaded videos to the uploader's channel. Previous work in this
  area has found that channel statistics are better predictors of a video's
  views than metadata about the video itself, so for a baseline prediction we
  will use the average of the views of up to five of that channel's most
  recently uploaded videos, and for an oracle we will use the candidate title as
a search term and return the view count of the search result whose channel view count
most closely matches the uploader's.}

\section{Overview}

The premise of this project revolves around the idea of predicting the view
count of YouTube videos given features like the title, thumbnail, description,
and similar information from previous videos uploaded on the same channel. We
had the initial idea of predicting views based just on a given title and
thumbnail and then using this predictor to generate video titles that are
optimized for maximizing views given a list of keywords that the title should
contain. However, previous work in this area found that the metadata of
individual videos were far less predictive of views than channel information
like subscriber count, channel view count, and channel video count. Therefore,
the input to our system will be a YouTube channel id, a title, and maybe a
thumbnail if time allows, and the output will be a prediction of the number of
views a video with that title and thumbnail would get if it were posted to that
channel. If time allows, we could use this view count predictor as a subsystem
of a title generator that given a channel id, thumbnail, and set of keywords,
will will generate a video title corresponding to those keywords that is
optimized for getting the most views for that channel.


\section{Previous Work}
\begin{enumerate}
  \item https://towardsdatascience.com/youtube-views-predictor-9ec573090acb
  \item https://github.com/allenwang28/YouTube-Virality-Predictor 
\end{enumerate}

\section{Datasets}
\begin{enumerate}
\item YouTube 8M dataset: https://research.google.com/youtube8m/
  \item Dataset from the existing virality predictor github page: https://github.com/allenwang28/YouTube-Virality-Predictor/tree/master/data 
\item https://channelcrawler.com 
\item YouTube Data API: https://developers.google.com/youtube/v3/ 
\end{enumerate}
\section{Approach}

\begin{enumerate}
\item Neural nets
  \item Using other pre-trained neural nets as feature extractors for images and titles
  \item Gradient boosting to comine multiple predictors into one
  \item Recurrent networks/LSTMs on titles
  \item Implement our own CNNs for feature extraction on thumbnails
\end{enumerate}


% Bibliography
%-----------------------------------------------------------------
\begin{thebibliography}{99}

\bibitem{Cd94} Author, \emph{Title}, Journal/Editor, (year)

\end{thebibliography}

\end{document}
