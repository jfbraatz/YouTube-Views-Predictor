%%%%%%%%%%%%%%%%%%%%%%%%%%%%%%%%%%%%%%%%%%%%%%%%%%%%%%%%%%%%%%%%%%%%%%%%%%%
%
% Template for a LaTex article in English.
%
%%%%%%%%%%%%%%%%%%%%%%%%%%%%%%%%%%%%%%%%%%%%%%%%%%%%%%%%%%%%%%%%%%%%%%%%%%%

\documentclass[12pt]{article}
\usepackage{fullpage,enumitem,amsmath,amsthm,amsfonts,amssymb,graphicx,float,listings}

% AMS packages:
% \usepackage{amsmath, amsthm, amsfonts}

% Theorems
%-----------------------------------------------------------------
\newtheorem{thm}{Theorem}[section]
\newtheorem{cor}[thm]{Corollary}
\newtheorem{lem}[thm]{Lemma}
\newtheorem{prop}[thm]{Proposition}
\theoremstyle{definition}
\newtheorem{defn}[thm]{Definition}
\theoremstyle{remark}
\newtheorem{rem}[thm]{Remark}

% Shortcuts.
% One can define new commands to shorten frequently used
% constructions. As an example, this defines the R and Z used
% for the real and integer numbers.
%-----------------------------------------------------------------
\def\RR{\mathbb{R}}
\def\ZZ{\mathbb{Z}}

% Similarly, one can define commands that take arguments. In this
% example we define a command for the absolute value.
% -----------------------------------------------------------------
\newcommand{\abs}[1]{\left\vert#1\right\vert}

% Operators
% New operators must defined as such to have them typeset
% correctly. As an example we define the Jacobian:
% -----------------------------------------------------------------
\DeclareMathOperator{\Jac}{Jac}

%-----------------------------------------------------------------
\title{CS 221 Project Proposal}
\author{Jon Braatz \& Lance Lamore}

\begin{document}
\maketitle

\abstract{We propose to create a model for predicting YouTube views for a video
  posted to an uploader's channel based on a canditate video title and statistics from
  previously uploaded videos to the uploader's channel. Time permitting, we will additionally incorporate thumbnails as a feature. Previous work in this
  area has found that channel statistics are better predictors of a video's
  views than metadata about the video itself, so for a baseline prediction we
  will use the average of the views of up to three of that channel's most
  recently uploaded videos, and for an oracle we will use the candidate title as
a search term and return the view count of the search result whose channel view count
most closely matches the uploader's.}

\section{Overview}

The premise of this project revolves around the idea of predicting the view
count of YouTube videos given features like the title, thumbnail, description,
and similar information from previous videos uploaded on the same channel. We
had the initial idea of predicting views based just on a given title and
thumbnail and then using this predictor to generate video titles that are
optimized for maximizing views given a list of keywords that the title should
contain. However, previous work in this area found that the metadata of
individual videos were far less predictive of views than channel information
like subscriber count, channel view count, and channel video count. Therefore,
the input to our system will be a YouTube channel id, a title, and maybe a
thumbnail if time allows, and the output will be a prediction of the number of
views a video with that title and thumbnail would get if it were posted to that
channel. If time allows, we could use this view count predictor as a subsystem
of a title generator that given a channel id, thumbnail, and set of keywords,
will will generate a video title corresponding to those keywords that is
optimized for getting the most views for that channel.


\section{Previous Work}
Another team of 4 people attempted a similar project[1]. They originally sought to
predict view counts based on an input image and title for a specific genre of
workout videos. It was based on an incorrect assumption that there would be enough 
similarities across the way top Youtuber's used thumbnails and wrote titles. This 
was not enough so they created another more complicated feature extractor and saw 
that 5 variables with the most influence over views on a video were subscriber 
count, channel video count, channel view count, previous video view count, and 
weeks published. Because none of these features seem specific to workout videos, we
hoped to expand this across all genres. 

\section{Datasets}
There are 2 data sets that we found with clean data to use[2][3]. Furthermore, we can use the Youtube data API [4]and another search called channel crawlers[5] which will list out channels based on a theme.  


\section{Approach}
We want to focus on extracting more 
features and use gradient boosting to combine multiple weak predictions into a more
powerful prediction using a neural net. We want to compile more comprehensive channel statistics since
those seemed to have the greatest influence. Time permitting we could incorporate pre-trained models to gather scores on the thumbnails like NSFW rating, and build another model to track the clickbait score of the title.  

\section{Baseline and Oracle}
The baseline for this project involves returning the average views of the most recent video on the channel. The Oracle involves searching across YouTube for a channel with similar size and returning the views of the of a similar themed video from a similar sized channel. The Oracle was difficult to design because the project involves predictions and because of this should not be treated as the ceiling for predictions. 


% Bibliography
%-----------------------------------------------------------------
\begin{thebibliography}{99}

\bibitem{Cd94} Aravind Srinivasan, \emph{YouTuve Views Predictor}, 

    \qquad https://towardsdatascience.com/youtube-views-predictor-9ec573090acb
\bibitem{Cd94}
YouTube 8M dataset: https://research.google.com/youtube8m/
\bibitem{Cd94}Dataset from the existing virality predictor github page: https://github.com/allenwang28/YouTube-Virality-Predictor/tree/master/data
\bibitem{Cd94}YouTube Data API: https://developers.google.com/youtube/v3/
\bibitem{Cd94}https://channelcrawler.com 


\end{thebibliography}

\end{document}
