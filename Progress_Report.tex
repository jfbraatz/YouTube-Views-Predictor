
\documentclass[12pt]{article}
\usepackage{fullpage,enumitem,amsmath,amsthm,amsfonts,amssymb,graphicx,float,listings}

\newtheorem{thm}{Theorem}[section]
\newtheorem{cor}[thm]{Corollary}
\newtheorem{lem}[thm]{Lemma}
\newtheorem{prop}[thm]{Proposition}
\theoremstyle{definition}
\newtheorem{defn}[thm]{Definition}
\theoremstyle{remark}
\newtheorem{rem}[thm]{Remark}

% Shortcuts.
% One can define new commands to shorten frequently used
% constructions. As an example, this defines the R and Z used
% for the real and integer numbers.
%-----------------------------------------------------------------
\def\RR{\mathbb{R}}
\def\ZZ{\mathbb{Z}}

% Similarly, one can define commands that take arguments. In this
% example we define a command for the absolute value.
% -----------------------------------------------------------------
\newcommand{\abs}[1]{\left\vert#1\right\vert}

% Operators
% New operators must defined as such to have them typeset
% correctly. As an example we define the Jacobian:
% -----------------------------------------------------------------
\DeclareMathOperator{\Jac}{Jac}

%-----------------------------------------------------------------
\title{CS 221 Project Progress}
\author{Jon Braatz \& Lance Lamore}


\date{November 2018}

\begin{document}

\maketitle

\abstract{ We propose to create a model for predicting YouTube views for a video
  posted to an uploader's channel based on a candidate video title and statistics from
  previously uploaded videos to the uploader's channel. To start with, we are working with
  a dataset of composed of 6623 random channels from youtube. From this list of channel, we created a simple feature vector for a video based on the channel statistics and began running a simple linear regression on the features in relation to the views of the video.}
  
\section{Creating Features}
\subsection{Jon's Edits}
We found a random sample of YouTube channel identifiers, and then used the
YouTube API to pull both coarse channel statistics like total videos uploaded,
total subscribers, and total channel view count, in addition to statistics
related to a subset of videos posted to that channel. The end goal of the
project is to predict the views of a candidate video based on title and other
channel statistics, so in order to choose a subset of videos with the most
that predict the views of a new candidate video most effectively, we should
choose a subset of videos from the channel that consists of the ``closest''
videos to the candidate according to some similarity metric. Two similarity
metrics that we thought we would most effective are based on the publish time of the candidate video
and its textual information like title and description. 

The efficacy of our method is determined largely by which subset of
video statistics we choose to use. We will want to be using the statistics of
the most similar videos on the channel to generate a prediction for the views in
a candidate video, so our measure of ``video similarity'' will determine our
model.

\section{Design of the model}

Our project is a regression task that, when given a candidate YouTube
video title and channel to upload it to, predicts the number of views a video
uploaded to the channel with that title would likely get. The statistics that
will be used to predict the view count will be the video statistics available
through the YouTube API of a subset of videos previously uploaded to the
provided channel. In particular, these statistics are:
\begin{enumerate}
  \item \texttt{ViewCount}
  \item \texttt{CommentCount}
  \item \texttt{LikeCount}
  \item \texttt{DislikeCount}
  \item \texttt{FavoriteCount}
  \end{enumerate}
The subset of videos to choose these statistics from will be the k-nearest
neighbors of the candidate video according to some similarity metric.
We chose two similarity metrics, one based on the video timestamps by considering
the statistics of the 10 most recently uploaded videos, and another based on
cosine similarity of the uploaded video titles to the candidate title.
The feature vectors are the same for both
models, and they are the video statistics of the ``most similar'' videos in the
channel to the candidate video, the model just differs in which videos those are.
We use a gradient boosted tree regressor to turn these statistics into a view
count prediction, and we use the \texttt{xgboost} machine learning library to do
so.

To train the regressor, we needed a dataset of ``candidate'' videos and
channels that they'd be uploaded to. To gather this data, we scanned through a
currently existing dataset with randomly sampled YouTube channel identifiers,
and using the YouTube search API we could pull statistics for all videos in
those channels. For our training dataset, we chose the most recently
uploaded video to each channel to act as a ``candidate'' video for that channel,
and then pulled the statistics of the 10 most similar videos to that one, both
in terms of publishing date and similarity of title according to cosine similarity.
Using the true view count of the most recently uploaded videos along with these
statistics, we could train our gradient boosted regressor.

\section{Algorithms Used}
Our problem is fundamentally a regression problem, and since we don't want to
limit ourselves to linear models, we decided to use gradient boosted regression.
We'll define a couple of terms first.

\begin{enumerate}
\item Regression: Regression consists of predicting a target value (in our case,
  the view count of a candidate video) based on several numerical or categorical
  predictors. In our case, the predictors are video statistics for the 10 videos
  in the channel that are most similar to the candidiate video according to some
  similarity metric.
\item Decision Tree: A decision tree is a method of regression that is built
  top-down from a root node and involves partitioning the data into subsets that
  contain instances with similar values. At each interior node of the tree, we
  apply a test to one of the predictor variables, and depending on the value of
  the outcome of the test we either go to the left or right subtree. Predictions
  of the target variable correspond to leaves in the tree. A decision tree
  regression algorithm will construct a decision tree based on training data
  that minimizes the least square error of the predictions and training target
  values, subject to a regularization term.
\item Gradient Tree Boosting: Gradient tree boosting is based on the idea of
  creating a potentially weak regression tree on the data, creating a regression
  tree for the residuals of the first model, and then sequantially applying this
  idea to create a strong regressor out of weak regressors.
\end{enumerate}
We use gradient boosted regressors from two libraries for our project, one from
\texttt{sklearn} and the other from \texttt{xgboost}. The first library offers
an easy way to visualize the feature importances that are learned by the
regressor, and the second offers better performance when predicting new values.
Also, since our target variable, view count, ranges over many orders of
magnitude, it is better to predict the logarithm of the view count rather than
the view count itself.

\section{Preliminary results}

to be wrttien after we can get a working regressor model. Compared the results to the oracle and baseline.  Quick analysis of why the results are what the are, why they are better or worse than expected.


\section{Future Work}

Moving forward, we would like to introduce more features. Specifically from the previous research conducted, we would like to include the coarser channel statistics as those had a large impact on the views a video would receive. 

\begin{thebibliography}{99}

\bibitem{Cd94} Aravind Srinivasan, \emph{YouTuve Views Predictor}, 

    \qquad https://towardsdatascience.com/youtube-views-predictor-9ec573090acb
y\bibitem{Cd94}YouTube Data API: https://developers.google.com/youtube/v3/



\end{thebibliography}

\end{document}
